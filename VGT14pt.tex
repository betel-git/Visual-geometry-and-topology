\documentclass[a4paper, 12pt]{article}

\usepackage{cmap}
\usepackage[T2A]{fontenc}
\usepackage[english, russian]{babel}
\usepackage[utf8]{inputenc}
\usepackage[left=2cm,right=1.5cm,top=2cm,bottom=2cm]{geometry}
\tolerance = 1000

\usepackage{amsmath, amssymb, amstext, amsfonts, amsthm}
\usepackage{mathtools}
\usepackage{etoolbox}
\usepackage{booktabs}
\usepackage{indentfirst}
\usepackage{enumerate}
\usepackage{nicematrix}

\usepackage{import}

\usepackage{graphicx}
\usepackage{tikz}
\usepackage{pgfplots}
\usepackage{float}
\usepackage{subcaption}

\usetikzlibrary{shapes.geometric}
\pgfplotsset{width=10cm,compat=1.18}
\usepgfplotslibrary{colorbrewer}

\newcommand{\N}{\mathbb{N}}
\newcommand{\Z}{\mathbb{Z}}
\newcommand{\R}{\mathbb{R}}
\newcommand{\Q}{\mathbb{Q}}
\newcommand{\CC}{\mathbb{C}}
\newcommand{\B}{\mathfrak{B}}
\newcommand{\Bo}{\mathring{B}}
\renewcommand{\phi}{\varphi}
\renewcommand{\epsilon}{\varepsilon}
\renewcommand{\emptyset}{\varnothing}
\renewcommand{\Gamma}{\varGamma}

\theoremstyle{plain}
\newtheorem*{theorem}{Теорема}
\newtheorem*{corollary}{Следствие}
\newtheorem*{statement}{Утверждение}
\newtheorem*{lemma}{Лемма}

\theoremstyle{definition}
\newtheorem*{definition}{Определение}
\newtheorem*{example}{Пример}

\theoremstyle{remark}
\newtheorem*{remark}{Замечание}

\usepackage{titlesec}
\titleformat{\section}{\LARGE \bfseries}{\thesection}{1em}{}
\titleformat{\subsection}{\Large\bfseries}{\thesubsection}{1em}{}
\titleformat{\subsubsection}{\large\bfseries}{\thesubsubsection}{1em}{}

\usepackage{hyperref}
\usepackage{xcolor}
\definecolor{linkcolor}{HTML}{225ae2}
\definecolor{urlcolor}{HTML}{225ae2}
\hypersetup{
    pdfstartview=FitH, 
    linkcolor=linkcolor,
    urlcolor=urlcolor,
    colorlinks=true
}

\usepackage{xifthen}
\usepackage{pdfpages}
\usepackage{transparent}

\newcommand{\incfig}[1]{%
    \def\svgwidth{\columnwidth}
    \import{./images/}{#1.pdf_tex}
}

%\title{Конспект курса «Наглядная геометрия и топология»}
%\author{\textbf{Автор курса:} профессор, д.ф.-м.н. Ведюшкина Виктория Викторовна 
%\and \textbf{Автор конспекта:} \href{https://github.com/betel-git}{Цыбулин Егор}, студент 108 группы}
%\date{\today}


\begin{document}

\begin{titlepage}
    \begin{center}
        \large Механико-математический факультет МГУ имени М.В. Ломоносова
        \vfill
        \Large Конспект курса «Наглядная геометрия и топология» \bigskip \\
        \large \textbf{Автор курса:} профессор, д.ф.-м.н. Ведюшкина Виктория Викторовна
        \textbf{Автор конспекта:} \href{https://github.com/betel-git}{Цыбулин Егор}, студент 108 группы
        \vfill
        Москва, \today
    \end{center}
\end{titlepage}
\tableofcontents
\newpage

% Дело вкуса — кто-то любит 14pt, кто-то — 12pt, поэтому ниже можно поменять стандартный 12pt на 14pt
\fontsize{14pt}{20pt}\selectfont

\import{chapters/}{c1.tex}
\import{chapters/}{c2.tex}
\newpage
\import{chapters/}{c3.tex}
\import{chapters/}{c4.tex}
\import{chapters/}{c5.tex}
\import{chapters/}{c6.tex}
\import{chapters/}{c7.tex}
\import{chapters/}{c8.tex}
\import{chapters/}{c9.tex}
\import{chapters/}{c10.tex}
\import{chapters/}{c11.tex}
\newpage
\import{chapters/}{c12.tex}
\import{chapters/}{c13.tex}

\newpage
\import{chapters/}{bibliography.tex}

\end{document}