% \documentclass[tikz,border=3mm]{standalone}
% \begin{document}
% \begin{tikzpicture}[
%     node/.style={circle, fill=black, inner sep=0pt}, % Стиль для точек
%     edge/.style={thick},
%     loop/.style={thick, out=225, in=135, looseness=50} % Стиль для округлой петли
% ]
%     % Узлы (точки)
%     \node[node] (A) at (0, 0) {};
%     \node[node] (B) at (2, 0) {};

%     \node[below] at (A) {A}; % Подпись для вершины A
%     \node[below] at (B) {B}; % Подпись для вершины B

%     % Кратные ребра
%     \draw[edge] (A) to[bend left=25] (B); % Первое ребро
%     \draw[edge] (A) to[bend right=25] (B); % Второе ребро

%     % Петля
%     \draw[loop] (A) to[loop] (A); % Округлая петля у вершины A
% \end{tikzpicture}
% \end{document}

\documentclass[tikz,border=3mm]{standalone}
\usetikzlibrary{graphs}

\begin{document}
\tikz \graph [multi] {
    a -- [bend left] b;
    a -- [bend right] b;
    a -- [loop] a;
};
\end{document}