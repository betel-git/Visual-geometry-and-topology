\chapter{Тест}
\section{(50 билет) Поверхности второго порядка}


\subsection{Общее уравнение}
\begin{definition}
    Поверхностью второго порядка называется множество точек трёхмерного аффинного или точечно-евклидова пространства, координаты которых в некоторой аффинной системе координат удовлетворяют уравнению $F(x, y, z) = 0$, где
    \[ F(x, y, z) = a_{11} x^2 + a_{22} y^2 + a_{33} z^2 + 2a_{12} xy + 2a_{13} xz + 2a_{23} yz + 2a_1 x + 2a_2 y + 2a_3 z + a_0 \]
    причём хотя бы одно из чисел $a_{11}, \ a_{22}, \ a_{33}, \ a_{12}, \ a_{13}, \ a_{23}$ отлично от нуля. Выражение $F(x, y, z)$ - \textit{многочлен второй степени} от переменных $x, y, z$.
    Уравнение $F(x, y, z) = 0$ называется \textit{общим уравнением} поверхности второго порядка.
\end{definition}

\begin{remark}
    Точно так же определяются повехности второго порядка в аффинном или точечно-евклидовом пространстве произвольной конечной размерности $n$;
    они задаются многочленами второй степени от $n$ переменных.
\end{remark}

Теория поверхностей второго порядка аналогична теории кривых второго порядка.


\subsection{Квадратичная часть и матрицы}
С каждым многочленом $F(x,y,z)$ связано \textit{квадратичное отображение} пространства (с данной системой координат) $f: A^3 \to \mathbb{R}$, которое каждой точке $X$ с координатами $(x, y, z)$ ставит в соответствие число $F(x,y,z)$.
Говорят, что это отображение представлено многочленом $F$ в данной системе координат.
В другой системе координат многочлен, представляющий ту же функцию, станет другим.

Как и в случае линий второго порядка:
\[ F(x,y,z) = \begin{pmatrix}
    x & y & z & 1
\end{pmatrix}
A
\begin{pmatrix}
    x \\ y \\ z \\ 1
\end{pmatrix}
=
\begin{pmatrix}
    x & y & z
\end{pmatrix}
A_1
\begin{pmatrix}
    x \\ y \\ z
\end{pmatrix} 
+ 2 \begin{pmatrix}
    x & y & z
\end{pmatrix}
\begin{pmatrix}
    a_1 \\ a_2 \\ a_3
\end{pmatrix} 
+ a_0,
\]
где
\[ A =
\begin{pmatrix}
    a_{11} & a_{12} & a_{13} & a_{1} \\
    a_{12} & a_{22} & a_{23} & a_{2} \\
    a_{13} & a_{23} & a_{33} & a_{3} \\
    a_{1} & a_{2} & a_{3} & a_{0} \\
\end{pmatrix}
\] – большая матрица,
\[ A_1 =
\begin{pmatrix}
    a_{11} & a_{12} & a_{13} \\
    a_{12} & a_{22} & a_{23} \\
    a_{13} & a_{23} & a_{33} \\
\end{pmatrix}
\] – малая матрица (квадратичной части).

\begin{definition}
    \[ F_1(x, y, z) = a_{11} x^2 + a_{22} y^2 + a_{33} z^2 + 2a_{12} xy + 2a_{13} xz + 2a_{23} yz\]
    называется \textit{квадратичной частью} многочлена $F$.
\end{definition}


\subsection{Закон изменения матриц при переходе к новой аффинной системе координат}
Дословно так же, как в случае линий, доказывается, что при переходе к новой системе координат матрицы $A$ и $A_1$ многочлена $F$, представляющие всё ту же функцию $f: A^3 \to \mathbb{R}$, меняются по закону $A_1^{'} = C^T A_1 C$ и $A^{'} = D^T A D$, где $A_1^{'}$ и $A^{'}$ – матрицы в новых координатах, $C$ – матрица перехода от старого базиса к новому (её столбцы – координаты новых базисных векторов в старом базисе),
$D = \begin{pNiceMatrix}\Block{3-3}<\Huge>{C}&&&x_0\\&&&y_0\\&&&z_0\\0&0&0&1\end{pNiceMatrix}$, где
$x_0, y_0, z_0$ - координата нового начала координат в старой системе координат.

В новой системе координат:
\[ F^{'}(x^{'},y^{'},z^{'}) = \begin{pmatrix}
    x^{'} & y^{'} & z^{'} & 1
\end{pmatrix}
A^{'}
\begin{pmatrix}
    x^{'} \\ y^{'} \\ z^{'} \\ 1
\end{pmatrix}
+ 2 \begin{pmatrix}
    x^{'} & y^{'} & z^{'}
\end{pmatrix}
A_1^{'}
\begin{pmatrix}
    a_1^{'} \\ a_2^{'} \\ a_3^{'}
\end{pmatrix} 
+ a_0,
\]
где буквы со штрихами – координаты, многочлен и матрицы в новой системе координат,

\[
\begin{pmatrix}
    a_1^{'} \\ a_2^{'} \\ a_3^{'}
\end{pmatrix} = C^T 
\begin{pmatrix}
    a_1 \\ a_2 \\ a_3
\end{pmatrix}.
\]
