\begin{theorem}[Сидлер]
    Пусть для двух равновеликих многогранников все инварианты Дена равны, то есть для любой аддитивной функции, определённой на их двугранных углах и числе $\pi$, выполнено $f(\pi) = 0, \ f(w_1) = f(w_2) \Rightarrow w_1, w_2$ равносоставлены.
\end{theorem}

\begin{definition}
    Два равновеликих многогранника $A, B$ называются равнодополняемыми, если существуют два равносоставленных многогранника $W_A, W_B$, для которых существуют разбиения $W_A = \bigcup_{i = 1}^k P_i \cup A, \ W_B = \bigcup P_i \cup B$, где $P_i$ — многогранники.
\end{definition}

\begin{theorem}[3-я проблема Гильберта]
    Существуют ли не равнодополняемые тетраэдры с равными основаниями и высотами?
\end{theorem}

\begin{statement}
    Если многогранники $A,B$ равнодополняемые, то их инварианты Дена равны.
\end{statement}
\begin{proof}
    $f(W_A) = f(W_B)$, где $$f(W_A) = f(A) + \sum_{i = 1}^{k} f(P_i) = f(B) + \sum_{i = 1}^{k} f(P_i) = f(W_B)$$
\end{proof}

Сюда тетраэдр Хилла.

\begin{statement}
    Тетраэдр Хилла, координатный тетраэдр не равносоставлены, а также не равнодополнены.
\end{statement}

\newpage
\section{Многообразия}

\begin{definition}
    Пусть $X$ — топологическое пространство. Пусть $B$ — семейство его открытых подмножеств такое, что любое открытое множество в $X$ есть объединение множеств из $B$. $B$ называется \textit{базой топологии}.
\end{definition}

\begin{definition}
    Хаусдорфово топологическое пространство называется \textit{двумерным ($n$-мерным) многообразием}, если оно имеет счётную базу, и у любой точки существует окрестность гомеоморфная открытому двумерному ($n$-мерному) диску.
\end{definition}

\paragraph{Классификация двумерных связных компактных многообразий}
Примеры:
\begin{enumerate}
    \item $\R^2$ — плоскость — не компактна.
    \item Сфера — связна, компактна, двумерное многообразие.
    \item Тор.
    \item Бутылка Клейна.
\end{enumerate}


Вроде всё, кроме девятой лекции (от 03.04.2025), здесь теперь расписано полностью. Когда доделаю лекцию, которая была в четверг — пока неясно, но выше есть хотя бы два определения оттуда. Всё, что сегодня (06.04.2025) было дописано, даже не перечитывалось, будьте осторожны! Буду, кстати, рад, если кто-нибудь отправит мне доказательство теоремы Сидлера, потому что я его не записал)).