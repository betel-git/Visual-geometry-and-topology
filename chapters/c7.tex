\begin{theorem}
    Для любого выпуклого многогранника выполнено \[B - P + \Gamma = 2\]
\end{theorem}
\begin{proof}
    
\end{proof}

\begin{definition}
    Правильный многогранник — это выпуклый многогранник, грани которого — это равные правильные многоугольники, все двугранные углы которого равны.
\end{definition}

\begin{remark}
    Все двугранные углы равны $\Leftrightarrow$ в любой вершине сходится одинаковое число рёбер.
\end{remark}

Почему можно заменить? Потому что верна следующая теорема:

\begin{theorem}[Коши]
    Дла выпуклых многогранника с одинаковым комбинаторным строением, имеющие равные соответствующие грани, совмещаются движением пространства (т.е. конгруэнтны).
\end{theorem}
\begin{proof}
    Без доказательства.
\end{proof}

\begin{theorem}
    В пространстве существует ровно пять правильных многогранников (платоновы тела).
\end{theorem}
\begin{proof}
    Все грани — правильные $n$-угольники и в любой вершине сходятся $m$ рёбер.

    \[\begin{cases}
        n \gamma = 2 P, \\
        m B = 2 P, \\
        B - P + \Gamma = 2.
    \end{cases} \Longrightarrow\]
    
    $$\frac{2 P}{m} - P + \frac{2 P}{n} = 2 \Rightarrow P \left(\frac{2}{m} + \frac{2}{n} - 1 \right) = 2$$
    $$\frac{2}{m} + \frac{2}{n} > 1$$
    при $m > 6$: $\frac{2}{m} \leq \frac{1}{3}$ и $\frac{2}{n} > \frac{2}{3} \Rightarrow n < 3$ — противоречие, т.к. если $n \geq 6 \Rightarrow m < 3$ — не бывает.
\end{proof}

Классификация многогранников:
\begin{enumerate}
    \item $n = 3, \ m = 3$: $P \left(\frac{2}{3} + \frac{2}{3} - 1 \right) = 2 \Rightarrow P = 6 \Rightarrow \Gamma = \frac{2  \cdot 6}{3} = 4$
\end{enumerate}

\begin{theorem}[Сабитов]
    При изгибании невыпуклого многогранника его объём сохраняется.
\end{theorem}
\begin{proof}
    Здесь могла быть ваша реклама.
\end{proof}

\begin{theorem}[Минковский]
    Пусть дан набор векторов $\overrightarrow{n_1}, \ \overrightarrow{n_k}$, никакие два из которых не сонаправлены и не лежат в одном полупространстве, и набор чисел $s_1, \dots, s_k$, что $\sum s_i \overrightarrow{n_i} = 0$. Тогда существует ровно один строго (вот тут дополни про фиктивные вершины "фиктивная вершина — вершина, в которой сумма плоских углов равна два пи") выпуклый многогранник, для которого вектора $s_i \overrightarrow{n_i}$ являются его ежом.
\end{theorem}
\begin{proof}
    Без доказательства.
\end{proof}