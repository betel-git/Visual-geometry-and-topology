\subsection{Теорема Жордана}
\begin{definition}
    Для любого подмножества $A$ плоскости отношение: «точки $P,Q \in A$ можно соединить непрерывной кривой, лежащей в $A$» является отношением эквивалентности. Соответствующие классы эквивалентности называются \textit{компонентами линейной связности множества $A$}.
\end{definition}

\begin{theorem}[Жордана (для ломаных)]
    Замкнутая вложенная ломаная разбивает плоскость на две компоненты связности. %(сюда рисунок №1)
\end{theorem}
\begin{proof}
    См. \cite{oshemkov}, сам допишу после следующей лекции или на выходных.
    Шаг 1. Число компонент $\leq 2$.
    \begin{lemma}
        Два радиуса разбивают круг на две компоненты.
    \end{lemma}
    \begin{proof}
        сюда рисунок 2 и 3
    \end{proof}
\end{proof}