\begin{statement}
    Непрерывный образ связного пространства связен.
\end{statement}
\begin{proof}
    $f: X \to Y$. От противного. Пусть образ несвязен. Тогда $Im f := f(X) = A \cup B$, где A, B — открытые и непустые множества, $A \cap B = \emptyset$. Рассмотрим прообразы $f^{-1}(A)$ и $f^{-1}(B)$, которые открыты в силу непрерывности отображения. Если множества не пересекаются, то и их образы не пересекаются (иначе существовала бы точка $x \in f^{-1}(A) \cap f^{-1}(B) \Longrightarrow f(x) \in A \cap B$, но $A \cap B = \emptyset$). Так как множества не пусты, то и их образы не пусты. $f^{-1}(A) \cup f^{-1}(B) = X \Longrightarrow X$ не связно — противоречие. 
\end{proof}

\begin{remark}
    Связность является топологическим инвариантом.
\end{remark}


\subsection{Линейная связность}
\begin{definition}
    \textit{Непрерывная кривая (параметрическая)} — непрерывное отображение ненулевого отрезка в топологическое пространство. $\gamma: [a,b] \to X$, где $\gamma$ непрерывна.
\end{definition}

\begin{example}
   $\gamma: [0, 2 \pi] \to \R^{2}$

    $\begin{cases}
        x = \cos{t}, \\
        y = \sin{t}, \\
        t \in [0, 2 \pi].
    \end{cases}$ 
\end{example}


\begin{definition}
    Топологическое пространство называется \textit{линейно связным}, если любые две его точки можно соединить кривой.
\end{definition}

\begin{example}
    $x, y$ — точки $X$, тогда $\exists \gamma: [\alpha, \beta] \to X: \ \gamma(\alpha) = x, \ \gamma(\beta) = y$
\end{example}

\begin{statement}
    Образ линейно связного пространства линейно связен.
\end{statement}
\begin{proof}
    % Композиция непрерывных отображений непрерывна:
    % $$\gamma: [\alpha, \beta] \to X, \ f: X \to Y.$$
    Пусть $y_1, y_2 \in f(X)$ — произвольные точки в образе пространства $X$. Тогда $$\exists x_1, x_2 \in X: \ f(x_1) = y_1, \ f(x_2) = y_2.$$
    Так как $X$ линейно связно, то существует непрерывный путь 
    \[\gamma [0,1] \to X,\]
    соединяющий $x_1, x_2$, то есть $\gamma(0) = x_1, \ \gamma(1) = x_2$.
Рассмотрим композицию
\[f \circ \gamma: [0,1] \to Y.\]
Эта композиция непрерывна как композиция непрерывных отображений и соединяет $y_1, y_2$, потому что 
\[(f \circ \gamma)(0) = f(\gamma(0)) = f(x_1) = y_1,\]
\[(f \circ \gamma)(1) = f(\gamma(1)) = f(x_2) = y_2.\]
Получаем, что для любых двух точек $y_1, y_2 \in f(X)$ мы нашли непрерывный путь в $f(X)$, их соединяющий. Следовательно, $f(X)$ линейно связно.
\end{proof}

\begin{statement}
    Если топологическое пространство линейно связно, то оно связно. (Наоборот, вообще говоря, неверно — как задачу можно попросить привести контрпример).
\end{statement}
\begin{proof}
    Пусть топологическое пространство линейно связно, но не связно. Тогда $X = A \cup B$ (данные открытые множества непусты и не пересекаются). Возьмём $x \in A, \ y \in B$. Пользуемся линейной связностью: $\exists \gamma: [0, 1] \to X$, $\gamma$ непрерывна, $\gamma(0) = x, \ \gamma(1) = y$.
    Заметим, что $$Im \gamma \cap A \neq \emptyset,$$ 
    $$Im \gamma \cap B \neq \emptyset,$$
    $$(Im \gamma \cap A) \cap (Im \gamma \cap B) = \emptyset,$$
    $$(Im \gamma \cap A) \cup (Im \gamma \cap B) = Im \gamma.$$
    Рассмотрим $Im \gamma$ как подпространство $X$ с индуцированной топологией. Тогда множества $Im \gamma \cap A, \ Im \gamma \cap B$ открыты в $Im \gamma$ и они образуют разделение $Im \gamma$ на два непустых непересекающихся открытых множества.

    Однако $Im \gamma$ — непрерывный образ отрезка $[0,1]$, а отрезок связен. Мы получили, что $Im \gamma$ несвязен, что противоречит теореме о том, что непрерывный образ связного пространства связен.
    % $Im \gamma \cap A$ — открыто в топологии образа $Im \gamma$, индуцированного топологии на $X$ (пользуемся топологией на подмножестве), $Im \gamma \cap B$ — открыто в топологии образа $Im \gamma$, индуцированного топологии на $X$ — получили противоречие с тем, что отрезок несвязен.
\end{proof}

\begin{example}
    В качестве контрпримера рассмотрим «топологическую синусоиду» (см.рис.\ref{fig:c2.2}):
    \[X = \left\{\left(x, \sin{\frac{1}{x}}\right) \ |\ x \in (0,1]\right\} \cup \left\{(0,y) | y \in [-1,1]\right\}.\]
    $X$ связно, однако не является линейно связным, потому что, например, не существует непрерывного пути из точки $(0,0)$ в $(\frac{1}{\pi}, 0)$.

    \begin{figure}[htbp]
        \centering
        \includegraphics[scale=1]{images/c2.2.pdf}
        \caption{«Топологическая синусоида».}
        \label{fig:c2.2}
    \end{figure}
\end{example}


\subsection{Компактность}
\begin{definition}
    Топологическое пространство \textit{компактно}, если из его любого открытого покрытия можно выбрать конечное подпокрытие.
\end{definition}

\begin{statement}
    Непрерывный образ компакта является компактом.
\end{statement}
\begin{proof}
    Пусть $f: X \to Y$ непрерывно. Покрываем образ открытыми множествами $\{U_{\alpha}\}$: $Im f \subseteq \bigcup\limits_{\alpha} U_{\alpha}$ — покрытие.

    $X \subset \bigcup\limits_{\alpha} f^{-1}(U_{\alpha})$ — открытое покрытие $X$ (т.к. $f$ непрерывно).

    $X \subset \bigcup\limits_{i = 1}^{n} f^{-1}(U_{i})$ — конечное подпокрытие (поскольку $X$ — компакт по условию).

    Тогда $Im f \subset \bigcup\limits_{i = 1}^n U_{i}$. Действительно, для любого $y \in f(X)$ существует $x \in X$ такой, что $f(x) = y$. Этот $x$ принадлежит некоторому $f^{-1}(U_{\alpha_i}) \Longrightarrow y \in U_{\alpha_i}$.
    %Пользуемся компактностью $X$: $Im f \subset \bigcup_{i = 1}^{n} f(U_{i})$
\end{proof}

\begin{remark}
    Компактность является топологическим инвариантом.
\end{remark}

\begin{statement}
    Замкнутое подмножество компакта есть компакт.
\end{statement}
\begin{proof}
    $M \subset X \subset Y$, $M$ замкнуто, $X$ компактно, $Y$ — топологическое пространство.

    $M \subset \bigcup\limits_{\alpha} U_{\alpha}$ — открытое покрытие $M$.
    $(Y \setminus M) \cup \bigcup\limits_{\alpha} U_{\alpha}$ — тоже открытое покрытие.

    Выберем в нём конечное подпокрытие:
    $X \subset (Y \setminus M) \cup \bigcup\limits_{i = 1}^n U_i$ — конечное подпокрытие.

    Тогда $M \subset \bigcup\limits_{i = 1}^n U_i$.
\end{proof}


\subsection{Хаусдорфовость}
\begin{definition}
    Топологическое пространство $X$ называется \textit{хаусдорфовым}, если у любых двух его различных точек существуют непересекающиеся окрестности.
\end{definition}

$\tau = \{X, \emptyset\} \Longrightarrow X$ не хаусдорфово.

\begin{lemma}
    Компакт $M$ в хаусдоровом пространстве $X$ является замкнутым множеством.
\end{lemma}
\begin{proof}
    Пусть $M \subset X$, $M$ — компакт;
    $x_0 \in X \setminus M$, $y \in M$.

    Пользуемся хаусдорфовостью: $x_0 \in U_{x_0}^y, \ y \in V_y, \ U_{x_0}^y \cap V_y = \emptyset$, где $U_{x_0}^y$ — открытая окрестность точки $x_0$, а $V_y$ — открытая окрестность точки $y$.

    $\bigcup_{y \in M} V_y$ — открытое покрытие всего множества $M$.

    Пользуемся компактностью: выберем конечное подпокрытие $M \subset \bigcup_{i = 1}^n V_{y_i}$, $\ y_i \in M$.
    $\bigcap_{i = 1}^n U_{x_0}^{y_i} = U$, $x_0 \in U$, $U \cap V_{y_i} = \emptyset$, $U$ открытое (конечное пересечение открытых — открыто) $\Longrightarrow X \setminus M$ открыто $\Longrightarrow M$ замкнуто.
\end{proof}

\begin{statement}
    $f: X \to Y$, $f$ — непрерывная биекция, $X$ — компакт, $Y$ — хаусдорфово топологическое пространство $\Longrightarrow$ $f$ — гомеоморфизм.
\end{statement}
\begin{proof}
    Пусть $f: X \to Y$, $X$ замкнуто, $M \subset X$, $M$ замкнуто $\Longrightarrow M$ компактно (по теореме о том, что замкнутое подмножество компакта есть компакт) $\Longrightarrow f(M) \subset Y$, где $f(M)$ тоже компактно (т.к. $f$ непрерывно) $\Longrightarrow f(M)$ замкнуто в $Y$ (по лемме, которая была выше). 
\end{proof}


\subsection{Фактор-топология}

% Фактор-топология: дано топологическое пространство $X$, и на нём задано отношение эквивалентности: $f: X \to X \setminus \sim$. $f$ сопоставляет каждой точке из $X$ её класс эквивалентности.
% Топология $X \setminus \sim$ задаётся отображением $f$.

\begin{definition}
    Пусть $X$ — топологическое пространство, а $\sim$ — отношение эквивалентности на $X$. \textit{Фактор-пространство} $X / \sim$ — это множество классов эквивалентности $[x]$ для всех $x \in X$. Топология на $X / \sim$ называется \textit{фактор-топологией}.

    Множество $U \subset X / \sim $ открыто в фактор-топологии тогда и только тогда, когда его прообраз $f^{-1}(U)$ открыт в $X$, где $f: X \to X / \sim$.
\end{definition}

%Важный пример не забудь добавить.
\begin{example}[нехаусдорфова пространства]
    Рассмотрим две числовые прямые $\R_1, \R_2$ и отождествим все их точки, кроме одной: $x \sim y \Leftrightarrow x = y, \ x \ne 0, x \in \R_1, \ y \in \R_2$. 
    Получили фактор-пространство $\R_1 \sqcup \R_2 / \sim$.
    Оно не является хаусдорфовым, так как у нулей числовых прямых нет непересекающихся окрестностей (см.рис. \ref{fig:c2.1}).

    \begin{figure}
        \centering
        \includegraphics{images/c2.1.pdf}
        \caption{Пример нехаусдорфова пространства}
        \label{fig:c2.1}
    \end{figure}
\end{example}
% конец второй лекции