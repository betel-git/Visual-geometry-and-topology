\section{Программа экзамена (2025 год)}
\begin{enumerate}
    \item Топологическое пространство. Примеры. Непрерывные отображения: эквивалентные определения.
    \item Основные классы топологических пространств: связные и линейно связные пространства. Их свойства.
    \item Основные классы топологических пространств: компактность, хаусдорфовость.
    \item Непрерывные отображения и гомеоморфизм. Примеры. Свойства непрерывных кривых.
    \item Комбинаторное и топологическое описание графов. Вложения графов. Существования вложения произвольного графа в трёхмерное пространство.
    \item Планарные графы. Теорема о существовании вложения планарного графа в плоскость так, чтобы его рёбра изображались ломаными.
    \item Теорема Жордана для замкнутой ломаной.
    \item Лемма о четырёх точках. Непланарность $K_{3,3}$.
    \item Теорема Жордана для замкнутой непрерывной кривой (доказательство, что кривая разбивает плоскость не менее чем на две компоненты).
    \item Формула Эйлера для плоских графов. Критерий Понтрягина-Куратовского.
    \item Определение выпуклой многогранной поверхности и многогранника. Выпуклые многогранники, три определения. Их эквивалентность.
    \item Формула Эйлера выпуклых многогранников. Классификация правильных многогранников. Теорема Коши о конгруэнтности выпуклых многогранников с одинаковыми гранями (без доказательства).
    \item Ёж многогранника и его свойства. Теорема Минковского (без доказательства).
    \item Теорема Бойяи-Гервина о равносоставленности равновеликих многоугольников.
    \item Инвариант Дена. Теорема о неравносоставленности равновеликих куба и правильного тетраэдра.
    \item Определение двумерного многообразия. Примеры двумерных многообразий (сфера, тор, бутылка Клейна, проективная плоскость) и их представление в виде склейки многоугольников.
    \item Теорема классификации двумерных компактных связных многообразий (без края). Представление в виде склеек многоугольников.
    \item Теорема классификации двумерных компактных связных многообразий (без края). Негомеоморфность многообразий, принадлежащих различным сериям и многообразий в одной серии с различными кодами. Эйлерова характеристика поверхности. Ориентируемость поверхности.
    \item Связная сумма поверхностей. Переформулировка теоремы классификации двумерных компактных связных многообразий (без края) в терминах связных сумм торов и проективных плоскостей.
    \item Регулярные параметризованные кривые на плоскости. Различные варианты определения гомотопии кривых. Примеры.
    \item Монотонная замена параметра регулярной кривой, поворот, параллельный перенос, гомотетия с положительным коэффициентом как примеры регулярной гомотопии замкнутых кривых.
    \item Число вращения. Теорема Уитни о классификации замкнутых регулярных кривых на плоскости с точностью до регулярной гомотопии.
    \item Классификация замкнутых регулярных кривых на сфере и цилиндре. Классификация замкнутых непрерывных кривых на торе.
    \item Гомотопическая эквивалентность. Примеры.
    \item Трёхмерные многообразия. Примеры. Склейка трёхмерной сферы и прямого произведения двумерной сферы на окружность из двух полноторий.
    \item Биллиарды. Изоэнергетическая поверхность. Класс гомеоморфности поверхности в зависимости от области биллиарда.
    \item Интегрируемые биллиарды. Биллиард в круге и прямоугольнике. Регулярные поверхности уровня пары интегралов.
\end{enumerate}
