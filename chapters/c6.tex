\begin{proof}
    Начав с произвольной вершины. пройдём по рёбрам графа, не проходя ни по какому ребру дважды. Мы не сможем сделать следующий шаг только в двух случаях: либо мы вернёмся в вершину, где уже были (это будет означать, что в графе есть цикл), либо вернёмся в вершину степени 1.

    Определим две операции:
    \begin{enumerate}
        \item Если в графе есть вершина степени 1, то удалим её вместе с ребром, которому она принадлежит.
        \item Если в графе есть цикл, то удалим любое ребро из этого цикла, не удаляя вершин, которым принадлежит это ребро.
    \end{enumerate}

    Мы можем выполнять эти операции, пока у графа есть рёбра. Значит, процесс остановится только тогда, когда граф состоит из одной вершины и не имеет рёбер (а для этого графа соотношение \eqref{eyler} выполнено).

    Осталось понять, что при выполнении вышеуказанных операций число $B - P + \Gamma$ не меняется.

    Первая операция: \[B \to B - 1\] \[P \to P - 1\]
    Вторая операция: \[B \to B\] \[P \to P - 1\]
    
    Отметим, что при выполнении обеих операций число $\Gamma$ не увеличивается (очевидно, что если некоторые точки можно было соединить непрерывной кривой, не пересекая рёбра графа, то после удаления ребра их можно будет соединить той же кривой)...

    Продолжение будет после коллокивума по математическому анализу. Снова можете посмотреть в \cite{oshemkov}.
\end{proof}

\begin{theorem}[Критерий Понтрягина-Куратовского]
    Граф планарен тогда и только тогда, когда он не содержит подграфов, гомеоморфных $K_5$ и $K_{3,3}$.
\end{theorem}
\begin{proof}
    Без доказательства.
\end{proof}

\section{Многогранники}
\begin{definition}
    \textit{Многоугольник (на плоскости)} — множество точек, ограниченное замкнутой вложенной конечнозвенной ломаной (вместе с этой ломаной).
\end{definition}

\begin{definition}
    Два многоугольника, расположенных в пространстве, называются \textit{смежными по ребру $a$}, если $a$ — их общее ребро. %и это отношение симметрично (т.е. если $K$ смежен по ребру $a$ с $M$, то и $M$ смежен по ребру $a$ с $K$).
\end{definition}

\begin{definition}
    \textit{Многогранная поверхность} — это конечный набор многоугольников в $\R^3$ такой, что для любого ребра любого многоугольника существует единственный другой многоугольник, смежный с ним по данному ребру (причём это отношение симметрично).
\end{definition}

\begin{definition}
    Многогранная поверхность называется \textit{вложенной}, если выполнены следующие условия:
    \begin{enumerate}
        \item Внутренние точки граней принадлежат только этим граням.
        \item Внутренние точки рёбер принадлежат только тем двум граням, которые смежны по данному ребру.
        \item У любой вершины существует «обход»: все грани, соответствующие данной вершине (как точке в $\R^3$) таковы, что для любых двух граней существует цепочка граней, их соединяющая. Причём все они смежные по рёбрам, инцидентных данной вершине.
        %\item Для любой вершины все грани, которым она принадлежит, образуют замкнутую цепочку граней, смежных по рёбрам, содержащих эту вершину. 
    \end{enumerate}
    То есть, у любой точки существует окрестность, гомеоморфная двумерному диску.
\end{definition}

\begin{definition}
    Многогранная поверхность \textit{связна}, если для любых двух граней существует цепочка граней, смежных по ребрам, их соединяющая.
\end{definition}

\begin{remark}
    Мы будем рассматривать только связные и вложенные многогранные поверхности.
\end{remark}

\begin{definition}
    Пусть дана вложенная связная многогранная поверхность $L$. Компактная часть пространства, ограниченная $L$ вместе с поверхностью $L$ называется \textit{многогранником}.
\end{definition}

\begin{definition}
    Множество в $\R^n$ называется \textit{выпуклым}, если для любых двух точек в нём, оно содержит весь отрезок между ними (отрезок, их соединяющий).
\end{definition}

\begin{definition}[1]
    Многогранник \textit{выпуклый}, если его множество точек выпукло.
\end{definition}

\begin{definition}[2]
    Многогранник \textit{выпуклый}, если он лежит в одном полупространстве, образованном плоскостью, содержащем любую его грань.
\end{definition}

\begin{definition}[3]
    Многогранник \textit{выпуклый}, если он совпадает (как множество в $\R^3$) с выпуклой оболочкой его вершин (выпуклая оболочка множества — это минимальное выпуклое множество, его содержащее).
\end{definition}

\begin{theorem}
    Определения (1)-(3) эквивалентны.
\end{theorem}
\begin{proof}
    $(1) \Longrightarrow (2)$ Тоже будет после коллокивума по математическому анализу.
\end{proof}