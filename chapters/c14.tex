\subsection{Классификация гладких замкнутых регулярных кривых на цилиндре с точностью до регулярной гомотопии}
Цилиндр можно задать следующими способами:
\begin{enumerate}
    \item Цилиндр в $\R^3$: $x^2+y^2=R^2$
    \item Плоскость со склеенными точками: зададим отношение эквивалентности 
    \[(\phi, z) \sim (\phi + 2\pi k, z), k \in \Z,\] тогда $\R^2 \setminus \sim$ — факторпространство и 
    \[(x_1, y_1) \sim (x_2, y_2) \Longleftrightarrow \begin{cases}
        y_1 = y_2,\\
        \frac{x_1-x_2}{2\pi} \in \Z.
    \end{cases}
    \]
    \item Плоскость без точки: $\R^2 \setminus \{\cdot\}$ — так как цилиндр гомеоморфен плоскости без точки.
\end{enumerate}

\textbf{Определим инварианты для кривых на цилиндре:}

Аналогично с плоскостью определяется число оборотов вектора скорости $R(\gamma)$.

Определим радиус-вектор кривой $\gamma(t)$ на плоскости, но через функции, которые задают кривую в полярной системе координат:
\[\gamma(t) = (\rho(t) \cos{\phi(t)}, \rho(t) \sin{\phi(t)}).\]
Причём пусть кривая не проходит через некоторую точку $A$, которую примем за начало координат. В таком случае $\rho(t) > 0$, а $\phi(t)$ определяется по модулю $2\pi$, где $t \in [0,T]$.
Тогда определим число оборотов вокруг точки $A$:
\[C(\gamma) = \frac{\phi(T) - \phi(0)}{2\pi} \in \Z.\]

Теперь число оборотов вокруг цилиндра можно считать как количество оборотов вокруг выколотой точки из третьего варианта задания цилиндра.

\begin{theorem}[аналог теоремы Уитни для цилиндра]
    Две гладкие замкнутые регулярные кривые на цилиндре регулярно гомотопны тогда и только тогда, когда их числа $R(\gamma)$ и $C(\gamma)$ совпадают, то есть
    \[R(\gamma_0) = R(\gamma_1), \ C(\gamma_0) = C(\gamma_1).\]
\end{theorem} 
\begin{proof}
    Доказательство сводится к приведению кривой с помощью регулярной гомотопии к хорошему виду\footnote{Я забыл сохранять исходники этих рисунков, поэтому на первых трёх рисунках начальная точка не склеивается в конечную, но вы уж как-то в голове представьте себе, что они на одном уровне :)}.

    Пусть нам дана замкнутая регулярная кривая, у которой векторы скорости в начальной и конечной точках совпадают. Вытянем у неё кусок возле конечной точки до начальной и один из концов получившейся петли выпрямим.
    
    \begin{figure}[ht]
        \centering
        \incfig[0.5\textwidth]{c14.3}
        \caption{Пример замкнутой регулярной кривой.}
        \label{fig:c14.3}
    \end{figure}

    \begin{figure}[ht]
        \centering
        \incfig[0.5\textwidth]{c14.4}
        \caption{С помощью регулярной гомотопии вытягиваем кривую.}
        \label{fig:c14.4}
    \end{figure}

    \begin{figure}[ht]
        \centering
        \incfig[0.5\textwidth]{c14.5}
        \caption{С помощью регулярной гомотопии вытягиваем до начальной точки.}
        \label{fig:c14.5}
    \end{figure}

    \begin{figure}[ht]
        \centering
        \incfig[0.5\textwidth]{c14.6}
        \caption{С помощью регулярной гомотопии вытягиваем одну из ветвей вверх горизонтально оси абсцисс.}
        \label{fig:c14.6}
    \end{figure}

    Это регулярная гомотопия, потому что $\gamma \sim (\alpha \cup \beta)$, где $\alpha$ — нижний кусок оранжевой кривой, $\beta$ — верхний горизонтальный кусок кривой.

    Осуществив такую гомотопию для $\gamma_0$ и $\gamma_1$, получим
    \[\gamma_0 \sim (\alpha_0 \cup \beta_0),\]
    \[\gamma_1 \sim (\alpha_1 \cup \beta_1),\]
    где $\alpha_0, \alpha_1$ — замкнутые регулярные кривые на плоскости, которые имеют одинаковые числа вращения (т.к. числа вращения $\gamma_0, \alpha_0$ и $\gamma_1, \ \alpha_1$ совпадают, а числа вращения $\gamma_0, \gamma_1$ одинаковы в силу нашей регулярной гомотопности); $\beta_0, \beta_1$ — одинаковые горизонтальные отрезки, имеющие длину $2\pi k$, где $k = C(\gamma_0) = C(\gamma_1)$.

    Теперь можно построить регулярную гомотопию между $\gamma_0$ и $\gamma_1$ — горизонтальные участки совпадают, а для нижних участков можно применить теорему Уитни на плоскости.

%     \begin{figure}[ht]
%     \centering
%     \begin{subfigure}[b]
%         \incfig[0.5\textwidth]{c14.3}
%         \caption{Пример замкнутой регулярной кривой.}
%         \label{fig:c14.3}
%     \end{subfigure}
%     \hfill
%     \begin{subfigure}[b]{0.48\textwidth}
%         \incfig[0.5\textwidth]{c14.4}
%         \caption{С помощью регулярной гомотопии вытягиваем кривую.}
%         \label{fig:c14.4}
%     \end{subfigure}
%     \hfill
%     \begin{subfigure}[b]{0.48\textwidth}
%         \incfig[0.5\textwidth]{c14.5}
%         \caption{С помощью регулярной гомотопии вытягиваем до начальной точки.}
%         \label{fig:c14.5}
%     \end{subfigure}
%     \hfill
%     \begin{subfigure}[b]{0.48\textwidth}
%         \incfig[0.5\textwidth]{c14.6}
%         \caption{С помощью регулярной гомотопии вытягиваем одну из ветвей вверх горизонтально оси абсцисс.}
%         \label{fig:c14.6}
%     \end{subfigure}
%     \caption{Последовательность гомотопий кривой.}
%     \label{fig:quad}
% \end{figure}
    
\end{proof} 

\subsection{Классификация гладких замкнутых регулярных кривых на торе с точностью до регулярной гомотопии}

С тором поступим также, как с цилиндром: зададим факторпространство следующим образом:
\[(x,y) \sim (x + 2\pi k, y + 2\pi m), \ k,m \in \Z.\]

Также зададим числа вращения:
\[R_1 = \frac{x(T) - x(0)}{2\pi},\]
\[R_2 = \frac{y(T) - y(0)}{2\pi}.\]

\begin{theorem}
    Две замкнутые кривые на торе гомотопны тогда и только тогда, когда их числа $R_1$ и $R_2$ совпадают.
\end{theorem} 

\newpage
\section{Прочие сюжеты}
\subsection{Гомотопическая эквивалентность}
\begin{definition}
    Пусть $X,Y$ — топологические пространства, $f,g$ — непрерывные отображения
    \[f: X \to Y,\]
    \[g: X \to Y,\]
    тогда $f,g$ называются \textit{гомотопными}, если существует непрерывное отображение 
    \[F: X \times [0,1] \to Y,\]
    \[F(x,0) \equiv f, \ F(x,1) \equiv g.\]
\end{definition} 

\begin{remark}
    Отношение гомотопности является отношением эквивалентности на множестве всех непрерывных отображений из $X$ в $Y$ (вот тут я, честно говоря, не помню, было ли это на лекции, но вообще данное утверждение является логической связкой между двумя определениями).
\end{remark}

\begin{definition}
    Топологические пространства $X,Y$ называются \textit{гомотопически эквивалентными}, если существует непрерывное отображение 
    \[f: X \to Y,\]
    \[g: Y \to X,\]
    \[f \circ g: Y \to Y,\]
    \[g \circ f: X \to X.\]
\end{definition} 

\begin{statement}
    Если $X$ и $Y$ гомеоморфны, то они гомотопически эквивалентны ($x \simeq y \Longrightarrow x \sim y$).
\end{statement} 

\begin{example}
    Замкнутый шар $B^2$ и открытый шар $D^2$ гомотопически эквивалентны точке, но попарно не гомеоморфны.
\end{example}

\begin{example}
    Окружность $S^1$ и кольцо $K = \{(x,y) \in \R^2 \ | \ 1 \leqslant x^2 + y^2 \leqslant 4\}$ гомотопически эквивалентны, но не гомеоморфны.
\end{example}

\begin{statement}
    $\R^n \sim \{\cdot\}$.
\end{statement} 
\begin{proof}
    \[f: \R^n \to \{\cdot\},\]
    \[g: \{\cdot\} \to \{0\},\]
    \[f \circ g: \{\cdot\} \to \{\cdot\},\]
    \[g \circ f: \R^n \to \overrightarrow{0}\]
    — надо доказать, что отображение гомотопно тождественному отображению:
    \[F(x,s) = \overrightarrow{x}(1-s),\]
    \[F(x,s): \R^n \times [0,1] \to \R^n,\]
    \[s=0: \ F(x,s) = \overrightarrow{x} - id_{\R^n},\]
    \[s=1: \ F(x,s) = \overrightarrow{0}.\]
\end{proof} 

\subsection{Трёхмерные многообразия}
\begin{definition}
    %Трёхмерное многообразие — компактное связное хаудорфово пространство $M^3$, каждая точка которого имеет окрестность, гомеоморфную евклидову пространству $\R^3$.
    \textit{Трёхмерное многообразие} — это хаусдорфово топологическое пространство со счётной базой, в котором для каждой точки существует окрестность, гомеоморфная открытому трёхмерному шару.
\end{definition} 

\begin{remark}
    $\partial M^3$ — двумерное многообразие.
\end{remark}

\begin{definition}
    \textit{Полноторие} — прямое произведение $D^2 \times S^1$ двумерного диска и окружности, то есть трёхмерная фигура, ограниченная тором.
\end{definition} 

\begin{remark}
    Полноторие — это бублик, тор — это оболочка бублика.
\end{remark}

\begin{example}(трёхмерных многообразий).
    \begin{enumerate}
        \item $S^3$ ($x_1^2 + x_2^2 + x_3^2 + x_4^2 = 1$ в $\R^4$);
        \item $\R P^3$;
        \item $S^1 \times S^1 \times S^1 = T^3$;
        \item $S^2 \times S^1$.
    \end{enumerate}
\end{example}

\begin{statement}
    $D^2 \times S^1 \sim S^1$.
\end{statement} 
