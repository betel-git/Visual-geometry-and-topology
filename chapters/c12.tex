\section{Классификация замкнутых регулярных кривых на плоскости с точностью до регулярной гомотопии}

\begin{definition}
    \textit{Непрерывная кривая} — непрерывное отображение отрезка в плоскость: $$\gamma: [a,b] \to \R^2.$$
    $$\gamma: (x(t), y(t))$$ — параметрическая кривая ($t$ — параметр).
\end{definition} 

\begin{definition}
    \textit{Гладкая параметрическая кривая} $\gamma = (x(t), y(t))$, $\gamma: [a,b] \to \R^2$, где $x(t), y(t) \in C^{\infty}([a,b])$. То есть, гладкая параметрическая кривая — это кривая, которая должна задаваться гладкими функциями, если её записать в координатах.
\end{definition}

В нашем курсе мы рассматриваем параметрические кривые, поэтому кривые, которые задаются, например, следующими параметрами:
$$\begin{cases}
    x = \cos{t}, \\
    y = \sin{t}, \\
    t \in [0, 2 \pi].
\end{cases}
\begin{cases}
    x = \cos{2t}, \\
    y = \sin{2t}, \\
    t \in [0, 2 \pi].
\end{cases}$$
разные, хотя как множество точек на плоскости они одинаковые.

Сюда примеры на гладкость.


\begin{definition}
    \textit{Регулярная кривая} — гладкая параметрическая кривая $\gamma(t) = (x(t), y(t))$, для которой вектор $\gamma'(t) = (x'(t), y'(t))$ (вектор скорости кривой) всюду отличен от нуля.
\end{definition}

\begin{definition}
    Пусть $\gamma(t) = (x(t), y(t))$, а $x(t), y(t)$ — гладкие периодические функции с периодом $T$. Тогда $\gamma: \R \to \R^2$ — \textit{гладкая замкнутая кривая}.
\end{definition} 

\begin{definition}
    Две замкнутые гладкие регулярные кривые $$\gamma_0(t), \gamma_1(t): \ [0, T] \to \R^2$$ называются \textit{регулярно гомотопными}, если существует отображение $$\gamma(t,s): [0, T] \times [0,1] \to \R^2,$$ которое:
    \begin{enumerate}
        \item является гладким;
        \item $s = 0, \ \gamma(t,s) \equiv \gamma_0(t); \ s = 1,\ \gamma(t,s) \equiv \gamma_1(t)$;
        \item $\gamma(t,s)$ при фиксированном $s$ — гладкая замкнутая регулярная кривая.
    \end{enumerate}
\end{definition} 

\begin{definition}
    Две замкнутые непрерывные кривые $$\gamma_0(t), \gamma_1(t): \ [0, T] \to \R^2$$ называются \textit{гомотопными}, если существует отображение $$\gamma(t,s): [0, T] \times [0,1] \to \R^2,$$ которое:
    \begin{enumerate}
        \item является непрерывным;
        \item $s = 0, \ \gamma(t,s) \equiv \gamma_0(t); \ s = 1,\ \gamma(t,s) \equiv \gamma_1(t)$;
        \item $\gamma(t,s)$ при фиксированном $s$ — непрерывная замкнутая кривая.
    \end{enumerate}
\end{definition} 

\begin{definition}
    Две гладкие регулярные кривые $$\gamma_0(t), \gamma_1(t): \ [0, T] \to \R^2$$ называются \textit{регулярно гомотопными}, если существует отображение $$\gamma(t,s): [0, T] \times [0,1] \to \R^2,$$ которое:
    \begin{enumerate}
        \item является гладким;
        \item $s = 0, \ \gamma(t,s) \equiv \gamma_0(t); \ s = 1,\ \gamma(t,s) \equiv \gamma_1(t)$;
        \item $\gamma(t,s)$ при фиксированном $s$ — гладкая регулярная кривая.
    \end{enumerate}
\end{definition} 


Пусть $\gamma$ — гладкая регулярная кривая такая, что $\gamma: [0, T] \to \R^2$, $(x(t), y(t))$. $\bar{0} \neq v = (x'(t), y'(t)) = (\rho(t) \cos{\phi(t)}, \rho(t) \sin{\phi(t)})$, где $\rho(t) > 0$ (длина $v$).

\begin{definition}
    Число вращения для замкнутой регулярной кривой $\gamma$ — число $$R(\gamma) = \frac{\phi(T) - \phi(0)}{2 \pi} \in \Z$$
\end{definition}

\begin{theorem}[Уитни]
    Две гладкие замкнутые кривые на плоскости регулярно гомотопны тогда и только тогда, когда их числа вращения совпадают.
\end{theorem}

Пусть $\gamma(t)$ — гладкая регулярная кривая. $\gamma'(t) = (x'(t), y'(t))$ — ненулевой вектор скорости.

\begin{definition}
    Гладкая регулярная кривая называется натурально параметризуемой кривой, если длина её вектора скорости равна 1.
\end{definition} 

\begin{statement}
    Для любой гладкой регулярной кривой существует натуральная параметризация $\gamma(t)$.
\end{statement} 

$$S = \int_{0}^{t} ||\gamma'(u)|| \,du$$

\begin{statement}
    Пусть $\gamma(t)$ — гладкая регулярная кривая, $t(\tau)$ — монотонно возрастающая функция, тогда прямые $\gamma(t)$ и $\gamma(\tau):= \gamma(t(\tau))$ регулярно гомотопны.
\end{statement} 
\begin{proof}
    $\gamma(t,s) = \gamma\left((1-s)t + s \tau(t)\right)$ ...
\end{proof} 

\section{Заключение}
06.05.2025 10:32 Ошибки, жалобы, предложения, дополнения и прочее можете \href{https://t.me/egor_tsy}{мне} присылать. Дополнил доказательства второй главы.