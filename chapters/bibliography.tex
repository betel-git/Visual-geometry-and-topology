\begin{thebibliography}{2}
    \bibitem{boltyansky}
    В.Г. Болтянский, В.А. Ефремович. Наглядная топология. Москва: Наука, 1983. 160 с.

    \bibitem{billiards}
    Г.А. Гальперин, А.Н. Земляков. Математические биллиарды (бильярдные задачи и смежные вопросы математики и механики). Москва: Наука, 1990. 288 с.

    \bibitem{kozlov}
    К.Л. Козлов. \href{https://teach-in.ru/file/synopsis/pdf/introduction-to-topology-kozlov-M.pdf}{Введение в топологию.} Москва: teach in, электронное издание. 92 с.

    \bibitem{oshemkov}
    А.А. Ошемков. \href{https://teach-in.ru/file/synopsis/pdf/visual-geometry-and-topology-oshemkov-M.pdf}{Наглядная геометрия и топология. Лекции.} Москва: teach in, электронное издание. 185 с.
    
    \bibitem{oshemkovsem}
    А.А. Ошемков. \href{https://teach-in.ru/file/synopsis/pdf/visual-geometry-and-topology-seminars-oshemkov-M1.pdf}{Наглядная геометрия и топология. Семинары.} Москва: teach in, электронное издание. 68 с.

    \bibitem{thebest}
    А.А. Ошемков и др. Курс наглядной геометрии и топологии. Москва: ЛЕНАНД, 2015. 360 с.

    \bibitem{prasolov}
    В.В. Прасолов. Элементы комбинаторной и дифференциальной топологии. Москва: МЦНМО, 2004. 352 с.

    \bibitem{dfgm}
    Учебные материалы по наглядной геометрии и топологии от кафедры дифференциальной геометрии и приложений механико-математического факультета МГУ имени М.В. Ломоносова [Электронный ресурс]. URL: \href{http://dfgm.math.msu.su/ngit.php}{http://dfgm.math.msu.su/ngit.php} (дата обращения: 19.02.2025).

\end{thebibliography}